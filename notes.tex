%\chapter{22/10/2019}
%\section{Opportunistic routing in Wireless Sensor networks ORW}
%In netrorks where is difficult to keep a topology, then sending broadcast to
%all nodes is a good idea. The problem is to identify the nodes that are best
%to forward a message and no node should forward a packet if any other node is
%transmitting. It' called opportunistic because it's not really deterministic
%like hte previous algo. \newline
%2012 is the year when ORW was thought. \newline
%There are some problems, like it is prone to cycles.
%\section{Sync transmission}
%We have talked about this with A-MAC. The capture effect works independently
%from the content of the packet.\newline In practice the standard T should be
%half that because of frequency, but it has been shown that there is no problem
%also with this.
%\subsection{Glossy}
%There are two fundamental parameters:
%\begin{itemize}
%	\item Slot: must be enough to accomodate a transmission and have effect on
%	life time: if they are two big you spend all time in listening, and also
%	drifting.
%	\item Number of retransmission: this is the main configuration parameter. If
%	the counter is incremented above this value then it's reseted. 
%\end{itemize}
%We can notice from slide 47 that latency does not increase or decrease at the
%increasing of the number of retransmission since few packets are lost and they
%not influence.\newline
%\subsection{Design and deployment guidelines}
%Many concurrent transmitter and/or hopes is bad, because there is a higher
%probability that some nodes are out of synch and this leads to collision. Also
%with hop there is some delay that is introduced and also in this case there is
%a higher probability of collision. \newline
%A large power imbalance between concurrent transmitters is good.\newline
%Large packets are bad (127 bytes are already too many). Channel business is
%not a problem since we are working in synchronization. The problem is that the
%more bits, the higher the probability of a bit failing and corrupting the
%packet. Moreover this is bad because if we are on the verge of drifting, and
%one node is slower than the others, we are going to drift. \newline
%\subsection{LWB}
%The idea is that you schedule the entire network alternating phases where the
%network is active and some where the network is inactive. When is active there
%is an initial flood (controller in slide 50 corresponds to a Glossy flood) and
%tells all the other nodes what to do in the period. The other nodes to their
%own floods in isolation transmitting their data without collision (not sure
%you can talk about collision). There is a slot fro contention were the nodes
%in contention flood together to see if something reaches. The information
%gathered from this last flood are going to be used to reschedule the
%transmission.
%\subsection{Chaos}
%Network wide aggregation: for example we want to compute the average
%temperature in the network. Normally you would create the tree, collect the
%data and then compute the value, and if the other nodes needed to know the
%value, then you should have sent it back. \newline
%Chaos exploits Glossy's floods. Each packet is made of flags and payload.
%There are as many flags as the nodes in the network. Then when a node
%transmits its packet, when it gets received is retransmitted adding some
%information, for example if we wanted to check the maximum, we want to change
%the content in the payload and set the packet of the nodes resending it. Note
%that C send doesn't get to A because of the capture effect (since the packets
%from B and C are different there can't be constructive interference. \newline
%Since this is flooded, it gets reliably received from all the nodes. 
%\chapter{Crystal and Ultra low Power Wireless Sensor Networks}
%
%\chapter{29/10/2019}
%\chapter{Tunnel}
%\chapter{Real-world}
%\chapter{25/10/2019}
%Prossima lezione 26/11/2019 non ci sarà.\newline
%\chapter{Bluetooth low energy}
%Enhanchments worked specially on speed since no cable, but still people want
%good speed. \newline
%\section{Bluetooth clasic}
%Bluetooth works in the 2.4GHz band because it gives nice properties and is ISM
%band so no need to buy. \newline
%79 channles, each with 1MZh bandwith.\newline
%Uses a frequency-hopping spread spectrum: instead of transmitting on a given
%frequency, we continue to change frequency so to be more resilient to
%interference. To do this we hope from frequency to frequency in a coordinated
%way. \newline
%From version 1.2 bluetooth uses adaptive FHSS: instead of hopping blindly,
%there are mechanisms to understand which frequency are being used around and
%avoid them.\newline
%Bluetooth specifies the entire stack up to application, and even provides an
%API as a profile which give the possibility to support more specifically
%applications. \newline
%The original architecture was master-slave: the master effectively determines
%the schedule and it coordinates the slave by telling them when to transmit, so
%like TDMA.\newline
%There are 3 type of networks:
%\begin{itemize}
%	\item point-to-point: 1 master, 1 slave
%	\item piconet: star topology: 1 master 7 slaves. 
%	\item scatternet: interconnections of piconet. 
%\end{itemize}
%The standard did not specify a routing protocol for it.
%\section{Enhancements}
%Unacceptable for some application to have lower power rates. \newline
%\section{BLE}
%Reduced number of channel: 40 channels of 2MHz. Only 3 channels are used for
%device discovering, carefully chosen in order to avoid overlapping. This
%allowed a faster device discovering and a lower consumption since instead of
%32 channels, only 3 are used. \newline
%Even though the names are different, yet once the connection is established,
%master-slave connection is bilateral, but yet the master choose the timing.
%
%\chapter{02/12/2019}
%\section{Birthday protocol}
%We can also choose $n$ and $k$ based on what we want our duty cycle to be.
%\newline
%One problem with this is collision between messages. \newline
%Moreover, even if little, there is a possibility that the two nodes will
%discover each other since they are always picking different slots. So we would
%like to have something deterministic.
%\section{Disco}
%The node wakes up every $P1$, $P_2$ prime numbers.\newline
%we are guarantee that they meet after the product of the two primes. \newline
%One problem is that, since we want to match a certain duty cycle, and since
%prime numbers are not uniformally distributed, then it's not guaranteed that
%we can achieve every duty cycle.
%\section{U-connecy}
%Based on primes, but only 1 per node (usually the same per all
%nodes).\newline
%A node can be in discovery mode and so wake up every $P$ and stay awake for 1
%slot, or in transmission mode, which happens every $P^2$ slots and lasts for
%$\frac{p+1}{2}$, that is it stays awake for half of the period. \newline
%This shows better performance than Disco, but still there are problems with
%prime numbers. \newline
%Probability method have very good avarage discoverying channels pretty fast,
%but some may never get discovered. With prime-based we can for sure get all
%channles, but we don't have a good latency on average as probabilitstic.
%\section{Searchlight}
%This protocol have both components: imagine to have two nodes with the same
%duty cycle. \newline
%The protocol says to choose a fixed period $t$, which does not need to be
%prime, then we fix a slot said anchor slot and then. But keeping the node up
%for $t/2$ is not good, then we move the probe slot around in two ways:
%\begin{itemize}
%	\item Sequential: the slot in the probe period is +1 with respect to the one
%	before. We are guaranteed to overlap in $\frac{t^2}{2}$ slots. A yellow on
%	yellow is still a match. 
%	\item Random probing: we pick one of the probe slots at random. I still can
%	move around, but the pattern is not deterministic. 
%\end{itemize}
%Every slots send and advertisment every end of the slot, but one coverage is
%enough, so we could half them. This though won't work if the slot are very
%aligned: just enlarge the slots of something.\newline
%R and S stands for random and sequential. Striped is the one with less
%beacons. 
%\section{BLEnd}
%Many of this work is very abstract. \newline
%Does not take into account not even collision between nodes.\newline
%The goal was to have predictable performance taking into account collisions
%and hardware limitations. \newline
%Slots are not fundamental, also BLE does not talk about slots. If we take this
%out we have a continuum, which allows us to not have bounds. \newline
%Moreover all this protocols assume a bidirectional discovery, A discovers B
%and viceversa, but this i snot necessary in many cases. 

