\chapter{Motion Planning}
\label{ch:motionPlanning}
In this thesis, we focuses on the aspect of motion planning considering the
equally important problem of mission planning as completed before starting the
motion planning task. While the former focuses on the best path to follow
starting from a position, executing the intermediate objectives and reaching
the final destination~\cite{motionplanning}, the latter focuses on the best way
of organizing the goals for each robots in the
environment~\cite{missionplanning}. The reason why mission planning is not
considered is due to the fact that usually warehouses use specialized software
to handle their internal structures, and such software is usually responsible
for the generation of an ordered set of goals. The aspect of motion planning is
particularly important in a populated environment because it needs to guarantee
people safety. 
%
%
%
\section{Safe Motion Planning}
In environments shared by both humans and robots, one of the main challenges is
safe motion planning. Firstly, a path planning algorithm for decentralized 
agents able to ensure collision avoidance was proposed in 2008, and it was 
proven by using reserved areas~\cite{purwin2008}. \acrf{PP} is one of the main 
techniques used to plan robotics paths for multiple agents. It was first
explored in 1987~\cite{erdmann1987} and it has then been refined providing a
generally applicable approach without limitations to the form or degrees of
freedom of the robots~\cite{vandenberg2005}. Later on in 2015, a revised
algorithm of PP was presented, proving that if a solution exists, it guarantees
to find a path for every robot to reach the final
destinations~\cite{michal2015}. At the same time, an asynchronous decentralized
algorithm for PP that has been proven to find paths for all the robots in the
system and to terminate twice as fast was also provided~\cite{michal2015}.
Another important factor for human-aware motion planning is the smoothing of
the trajectory which is done by interpolating the points with well-behaved
curves: we proposed a novel solution using dynamic programming for the 
multi-point Markov-Dubins problem~\cite{frego2020} in 2020 and we later refined
it using parallel programming on GPUs~\cite{saccon2021}. \newline
The integration of human motion in the planning has been studied for
single-agent motion in various works that can then be adapted to a multi-agent
scenario. A first reactive model based on velocity obstacles, i.e., a set of
velocities which the robot should not use to avoid a dynamic obstacle was
proposed in 1998~\cite{fiorini1998}. In 2020, a refinement was presented which
was meant to be applied to multi-agent navigation by using the velocity
obstacle paradigm only for local and reactive navigation, while using RTT* to
generate the global path of each robot, showing a more human-like and
collaborative behaviour of the robots~\cite{boldrer2020}. Another effective
solution was proposed in 2018 using a combination of velocity obstacles,
dynamic windows, that is, a set of velocities easily reachable by the robot
that would allow it to either avoid the obstacle or to stop, and Monte Carlo
sampling to account for the dynamic obstacles~\cite{claes2018}. Other solutions
are based on learning-based methods and predictive planners. The former suffer
from the generalization problem and hence produce solutions that are 
troublesome to deploy in sensitive environments due to the difficulty in 
providing safety guarantees. The latter instead produce human-aware paths if 
the models are well-fitted and can rely on precise estimations. Indeed, one of 
the main difficulties of multi-agent navigation in environments with the 
presence of humans is to find a good model for the human motion. Recent
works~\cite{bevilacqua2018}~\cite{conflict_detection} showed promising results
using headed social force model, i.e., social force model also considering the
direction of the person. In 2019, it has been proved that also the
maximum-entropy model was able to offer safe navigation, for which the human
dynamics are affected by an action drawn from a Boltzmann probability
distribution~\cite{bajcsy2019}.
%
%
%
\import{./}{1-SAPF}
\import{./}{2-MAPF}
