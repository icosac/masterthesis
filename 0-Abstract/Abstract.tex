\chapter{Abstract}
\label{ch:abstract}
The new industrial revolution is proving the critical role of robotics in
industry and also in the everyday life, allowing for considerable
improvements in various scenarios. In particular, in this thesis we will focus
on the motion planning for multiple robots by using a real warehouse located in
the center of Italy as the main scenario. \newline
The \acrf{MAPF} problem amounts to finding the best paths for a group of
robots to navigate an environment, which is usually modelled as a graph.
First, we start with a description of the state-of-the-art algorithms used to
solve the \acrf{SAPF} problem since they are used as a building block for many
algorithms solving the \acrs{MAPF} problem. After that we move to a precise
description of the algorithms that solve the main problem showing their
strengths and their weak spots. In particular, we focus on the description of
correct and optimal solvers, i.e., solvers that are guaranteed to return an
optimal solution. Next we propose a concise review of some variants of the
classical \acrs{MAPF} problem explaining which aspects are taken from each
variant to model our problem on the warehouse. \newline
Then, we describe the three algorithms we have chosen and implemented to tackle
the aforementioned problem explaining the modification we had to make to them
in order to be used in our scenario. Two of these
are based on the same \acrs{MAPF} algorithm, but they use different
internal solvers as it will be more clear later on. Finally the third algorithm
is based on constraint programming, which differs from all the other
implementations. \newline
After that, we describe the way random tests were generated in order to respect
some limitations of the problem and to produce a series of tests of increasing
difficulty. We also delineate a suite of tests taken from the warehouse map,
explaining which alterations had to be made. \newline
Next a discussion on the results and the future work that may come with it
is carried out. The results show the implemented algorithm's limitations, from
the analysis of which we describe a particularly difficult case. Eventually, we
propose some modifications to the algorithms that may improve the performance
and the number of cases they manage to solve.
