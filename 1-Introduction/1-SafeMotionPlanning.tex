\section{Safe Motion Planning}
In environments shared by both humans and robots, one of the main study
subjects is safe motion planning. Firstly, Purwin et al. (2008) proposed a path
planning algorithm for decentralized agents able to ensure collision avoidance,
proven by using reserved areas. Prioritized path Planning (PP) is one of the
main techniques used to plan robotics paths for multiple agents. It was first
explored by Erdmann et al. (1987) and it has then been refined by van den Berg
et al. (2005) providing a generally applicable approach without limitations to
the form or degrees of freedom of the robots. Later, Michal Čáp et al. (2015)
presented a revised algorithm of PP proving that if a solution exists, it
guarantees to find a path for every robot to reach the final destinations. The
authors also provided an asynchronous decentralized algorithm for PP that has
been proven to find paths for all the robots in the system and to terminate
twice as fast. Another important factor for human-aware motion planning is the
smoothing of the trajectory which is done by interpolating the points with
well-behaved curves. Frego et al (2020) proposed a novel solution using dynamic
programming for the multi-point Markov-Dubins problem and Saccon et al (2021)
refined the solution using parallel programming on GPUs. \newline
The integration of human motion in the planning has been studied for
single-agent motion in various works that can then be adapted to a multi-agent
scenario. Fiorini et al. (1998) proposed a first reactive model based on
velocity obstacles, i.e., a set of velocities which the robot should not use to
avoid a dynamic obstacle. Boldrer et al (2020) proposed a refinement of this
model to be applied to multi-agent navigation by using the velocity obstacle
paradigm only for local and reactive navigation, while using RTT* to generate
the global path of each robot, showing a more human-like and collaborative
behaviour of the robots. Another effective solution was proposed by Claes et al
(2018), who used a combination of velocity obstacles, dynamic windows, that is,
a set of velocities easily reachable by the robot that would allow it to either
avoid the obstacle or to stop, and Monte Carlo sampling to account for the
dynamic obstacles. Other solutions are based on learning-based methods and
predictive planners. The former suffers from the generalization problem,
resulting in troublesome solutions to be deployed in sensitive environments due
to the difficulty in providing safety guarantees. The latter instead produce
human-aware paths if the models are well-fitted and can rely on precise
estimations. Indeed, one of the main difficulties of multi-agent navigation in
environments with the presence of humans is to find a good model for the human
motion. Bevilacqua et al (2018) and Antonucci et al (2021) showed promising
results using headed social force model, i.e., social force model also
considering the direction of the person. Bajcsy et al (2019) proved that also
the maximum-entropy model was able to offer safe navigation, for which the
human dynamics are affected by an action drawn from a Boltzmann probability
distribution.

