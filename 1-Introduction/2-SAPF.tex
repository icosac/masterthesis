\section{Single-Agent Path Finding}
\label{sec:SAPF}
The \acrf{sapf} problem is the problem of finding the best path on a graph 
between two given nodes or vertexes. Such problem is of great importance in 
various scenarios. Indeed, one of the main algorithms used to solve the 
\acrs{sapf} problem, \astar, has been successfully applied to GPS localization
in order to improve the waypoints accuracy for remote controlled 
agents~\cite{astar_gps}. Nevertheless, the field in which \acrl{sapf} has found
the most importance is the field of robot routing and planning, as the problem 
name also suggests. The two problems differentiate mainly on the aspects that 
they consider: while a routing problem consider the topology of the 
environment, a planning problem only considers the relationships between 
positions. \commentE{Non sono sicuro di questa differenza, viene menzionata in
un articolo~\cite{robot_routing} però non ha molta sostanza.}
\acrs{sapf} algorithms have been successfully implemented in robot routing, 
where they have been used to search a graph constructed by environmental data
in order to avoid obstacles and to explore possible 
routes~\cite{robot_routing}. Moreover, they have also been adopted to run in 
more than 2 dimensions such as when running 
manipulators~\cite{robot_mani}. \newline
This work though focuses on the path planning problem that can be defined as 
follows:
\begin{definition}[Single-Agent Path Finding]
Given a graph $G = (V,E)$, where $V$ is the set of the vertexes and $E$ the set
of edges joining two vertexes, the \acrf{SAPF} problem consists in finding the
shortest feasible plan $\pi$ between a starting vertex $S\in V$ and a final one 
$F\in V$. 
\end{definition}
A plan $\pi$ is the sequence of $s$ actions that take the agent from the 
starting position $S$ to the final position $F$ in $s$ steps:
\[ \pi=\{a_1...a_s\} : a_s(...a_2(a_1(S))...)=F \]
Due to its definition, the \acrs{sapf} problem can be reduced to the problem of
finding the shortest path on a graph. What follows is a brief description of
the main algorithms that can be applied to \acrl{sapf} which can be divided in
deterministic (e.g. Dijkstra) and heuristic ones (e.g. \astar).
%
%
\subsection{Dijkstra}
The Dijkstra algorithm~\cite{dijkstra} is meant to find the shortest path 
between two nodes on a graph which edges have only positive values. Note that
the graph needs to be strongly connected, i.e., there must be at least one path
between any two nodes. While this seems quite a strong limitation, industrial
scenarios usually provide such graphs by the simple reasoning that no node can
be a sink since it must be possible for an agent to come back from each
location, that is, usually graphs modelled on warehouses are either undirected, 
and hence strongly connected, or directed but no node can be a sink. \newline
The work of Dijkstra published in 1959~\cite{dijkstra} presents two possible
algorithms, one to find the shortest path from one node to another and one to
find a tree of minimum length starting from a node and reaching all the other
nodes. This work focuses on the latter for which a brief description can be 
found looking at the pseudo code Algorithm~\ref{algo:dijkstra}.
\begin{algorithm}
  \DontPrintSemicolon
  \caption{Description of the Dijkstra algorithm}
  \label{algo:dijkstra}

  \KwIn{$G=(V,E)$ strongly connected graph}
  \KwIn{$S$ initial node}
  \KwIn{$F$ final node}\;

  \KwData{\texttt{dist} vector of distances from $S$}
  \KwData{\texttt{prev} vector of the nodes from which the path comes from}
  \KwData{\texttt{OPEN} set of nodes to be considered}\;

  \For{$v \in V$}{
    $dist[v]\gets inf$\;
    $prev[v]\gets NULL$\;
    $\texttt{OPEN}.add(v)$
  }
  $dist[$S$]=0$\;
  \;
  \While{$\texttt{OPEN}$}{
    $u \gets node from \texttt{OPEN} with smaller distance$\;
    $\texttt{OPEN}.del(u)$\;
    \;
    \For{v $\in neighbors(u)$ still in \texttt{OPEN}}{
      $tmp \gets dist[u] + E(u,v)$\;
      \If{tmp  < dist[v] and dist[u] != inf}{
        $dist[v]\gets tmp$\;
        $prev[v]\gets u$\;
      }
    }
  }
\end{algorithm}\newline
The complexity of the algorithm depends on the number of vertexes and edges.
Moreover different and improved versions of the algorithm have different
worst-case performance, but the initial one proposed by Dijkstra runs in time
$O((\abs{V}+\abs{E})\log\abs{V})$.\newline
Finally, the protocol has been successfully used in robot path
planning~\cite{dijkstra1}~\cite{dijkstra2}~\cite{dijkstra3}.
%TODO Add a picture of Dijkstra?
%
%
\subsection{\astar}
\astar is an heuristic best-first search algorithm for finding the shortest
path on a graph~\cite{MAPF_overview}. It is also an admissible algorithm, that
is, it is guaranteed to find an optimal from the starting node to the arrival
one~\cite{astar}. \newline
The idea of \astar is to direct the search over the nodes towards the arrival
node without having to necessarily examine all the vertexes. To do so, \astar
keeps a set of nodes to be visited, usually called \texttt{OPEN}, which is
initially set to only the starting node, but then it is added with the
neighbors that the algorithm deems worthy to be expanded. A node is said to be
expanded when it is added to the \texttt{OPEN} set to be analyzed later on.
\newline
The choice of which nodes should be expanded and which not, is given by the
heuristic function. Indeed, when examining the neighbors 
$u\in \texttt{neigh(n)}$ of the considered node, \astar uses a heuristic 
to estimate $h(u)$ to estimate the distance to the arrival vertex. Let $h*(u)$
be the perfect heuristic, that is, a function that returns the correct distance
from the node $u$ to the arrival vertex, then if $h*(u)$ is known for all the
nodes, the best path is obtained just by choosing to go to the neighbor with
the lower heuristic distance between neighbors. It has been proved that if
$h(n)\leq h*(n)$, then the heuristic is admissible and \astar is
optimal~\cite{astar}.



