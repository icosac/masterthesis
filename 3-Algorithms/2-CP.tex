\section{Constraint Programming}
As said, \acrf{MAPF} problem is NP-hard~\cite{lavalle} and so its possible to
model it using constraint programming. Such technique has been deeply explored
using different solvers for the constraints:
\begin{itemize}
  \item \acrf{CSP}: a set of logical and mathematical statements that must
    satisfy a number of constraints~\cite{CSP};
  \item \acrf{SAT}: a set of boolean formulas for which the solver tries to
    find a set of boolean variables such that the formulas are true, i.e.,
    satisfiable~\cite{SATSolver}. 
  \item \acrf{MIP}: a set of linear equations that must be solved while
    satisfying a list of constraints~\cite{MAPF_overview};
\end{itemize}
The first step was to take the model proposed by Bart\'{a}k et al. in 
2017~\cite{picat1} and implement it using CPLEX \acrf{OPL}~\cite{OPL}. The goal 
of moving the model to another programming language was to exploit the 
straightforward similarity between the problem and \acrs{MIP} and also to 
exploit the optimal performance of IBM's solver. \newline
Bart\'{a}k et al. used Picat, a logic programming language that allows the 
programmer to code a series of constraints as generally as possible and then to
solve these by means of different solvers offered by the framework. Indeed,
Picat can solve the coded constraints either with a \acrs{SAT} solver, or with
\acrs{MIP} or also with a \acrs{CSP} solver. This actually allows for an easy
and great comparison between the possible solvers, but in our study we focus
on having a reliable, easy-to-implement and especially easy-to-maintain solver.
Picat presents an initial steep learning curve, which does not comply with the
easy-to-implement and easy-to-maintain requirements. \newline
Instead, CPLEX has a much nice learning curve (largely but badly documented,
Ed.) with an easy to pick up idea of how the solver works and a clear
distinctions of the phases in which the programmer should divide their code.
This is particularly noticeable if one starts writing the code using 
\acrs{OPL}. A CPLEX program structure is usually divided in:
\begin{itemize}
  \item Pre-processing: we used this step to ensure that the input problem was
    well-defined;
  \item The input data: in \acrs{OPL} a \texttt{.dat} file is used to store the
    input variables;
  \item The objective function: CPLEX is able to minimize (or maximize) only
    one objective function, but for our use-case this is enough;
  \item A series of constraints to which the decision variables should be bound
    to;
  \item Post-processing: a series of instructions which conclude the program,
    usually used to print and check the solution. 
\end{itemize}
We are now going to describe this steps. 
%
%
\subsection{Variables}
A variable in CPLEX can either be an input variable or a \textit{decision
variable}, which is a variable whose value should be computed by the solver.
\newline
The input variables in our cases are the following: 
\begin{itemize}
  \item \milc{n_agents}: the number of agents in the problem;
  \item \milc{n_nodes}: the number of nodes in the graph;
  \item \milc{n_steps}: the number of steps in which the problem is supposed to
    be resolved;
  \item \milc{connect[n_nodes][n_nodes]}: a non negative connectivity matrix of
    the nodes;
  \item \milc{n_goals[n_agents]}: a vector containing the number of goals each
    agent has to complete;
  \item \milc{max_goals}: the maximum number of goals an agent must reach;
  \item \milc{init_pos[n_agents]}: a vector containing the initial positions of
    all the agents;
  \item \milc{end_pos[n_agents]}: a vector containing the final positions that
    each agent should reach;
  \item \milc{goal_pos[n_agents][max_goals]}: a list of lists of goal
    positions, one list for each agent. 
\end{itemize}
Notice that it is not possible to know \milc{n_steps} a priori, hence its value
is usually a simple estimation of what the actual solution would be. The value
is refined during the execution: whenever a model is not solvable, it is
incremented by one to check whether it is possible to solve the scenario with
more time. The value is incremented until a solution is found. \newline
To solve the \acrs{MAPF} problem we use multiple decision variables. The
primary one is the \milc{X[n_steps][n_nodes][n_agents]}, which is a boolean
three dimensional matrix that keeps track of whether an agent $a_i$ is on node
$n$ at time $t$. \newline
To keep track of whether the agents have reached or not their goals, another
three-dimensional matrix is created:
\milc{goal_points[n_agents][max_goals][n_steps]} that states whether an agent
$a_i$ has reached a certain goal $g$ at a certain moment $t$. \newline
Finally, we need to keep track of the movements of the agents so that we can
then minimize the objective function of choice. This is done by using another
matrix, \milc{movement[n_agents][n_steps]} that memorizes whether an agent
$a_i$ at time $t$ has moved or not. \newline
One final structure is needed to avoid swap conflicts, but will be discussed
later on.
%
%
\subsection{Objective Function}
The objective function that was initially sought was the minimization of the 
sum of the movements of all the agents in the graph, i.e., minimizing the 
\textit{sum of costs}~\cite{picat1}. To do this in CPLEX one could simply 
write:
\begin{minted}[linenos=false]{C}
minimize sum(a in agents, s in steps) movement[a][s];
\end{minted}
which corresponds to just minimizing the sum of all the cells in the movement
matrix. We will discuss later about the implementation in C++ of the model and
how it is possible to also minimize for the makespan. 
%
%
\subsection{Constraints}
The constrained defined by Bart\'{a}k et al.~\cite{picat1} are being reused:
\begin{itemize}
  \item Each agent $a$ cannot be on more than one node at each time step:
    \begin{equation}
      \forall s \in \mathcal{S},~\Sum_{n \in \mathcal{N}} X[s][n][a]=1
      \label{eq:picat1_one_agent_one_node}
    \end{equation}
  \item Each node $n$ cannot be occupied by more than one agent at each time
    step:
    \begin{equation}
      \forall s \in \mathcal{S},~\Sum_{a \in \mathcal{A}} X[s][n][a]\leq 1
      \label{eq:picat1_one_node_one_agent}
    \end{equation}
  \item If an agent $a$ occupies a node $n$ at time $s$, it will occupy a
    neighboring node $m$ at time $s+1$:
    \begin{equation}
      X[s][n][a]=1 \quad \Rightarrow \Sum_{m\in \neigh{n}} X[s+1][m][a]=1
      \label{eq:picat1_move_to_neigh}
    \end{equation}
\end{itemize}
The above constraints do not bind the initial and final position of the
agents. Indeed, in the code proposed in~\cite{picat1}, the assignment of
the initial and final position for each agent is done via preprocessing,
but this was not possible with \OPL since the decision variables cannot be
directly modified. Two constraints were then introduced:
\begin{equation}
  \forall a \in \mathcal{A},~X[t_0][S_a][a] = 1
  \label{eq:cplex_init_pos}
\end{equation}
\begin{equation}
  \forall a \in \mathcal{A},~X[t_f][E_a][a] = 1
  \label{eq:cplex_final_pos}
\end{equation}
%
\subsubsection{Swap Conflict}
Although the swap conflict was not taken into consideration due to the
increased computational cost as stated in~\cite{picat1}, in this work, it 
is important to avoid this type of conflict\footnote{We would like to make
known that also a constructive email exchange has been carried out with the 
original authors of the Picat algorithm, highlighting this aspect and a 
possible solution.}. A first constraint to avoid a swap conflict is the 
following:
\begin{equation}
  \begin{split}
  \forall s \in\mathcal{S}/\{t_f\},~\forall a_i \in\mathcal{A},~\forall 
  a_j\in \mathcal{A}/\{a_i\},~\forall n_i \in \mathcal{N},~\forall n_j \in 
  \mathcal{N}/\{n_i\},\\
  X[s][n_i][a_i] \wedge X[s][n_j][a_j] \Rightarrow
  X[s+1][n_j][a_i]\wedge X[s+1][n_i][a_j]
  \end{split}
  \label{eq:cplex_no_swaps_no_edges}
\end{equation}
This though increases a lot the time required to find a solution. \newline 
The best alternative seems to follow the work from Bart\'{a}k and \v{S}ancara
in 2019~\cite{picat2}, who propose to use another structure to keep track of
the edges occupation. For this reason the following three dimensional matrix is
used: \milc{edges[n_nodes][n_nodes][n_steps]} which keeps track of whether at
time step $s$ the edge from node $n_i$ to node $n_j$ is being used or not. 
\newline
The constrain is then so defined:
\begin{equation}
  \forall s \in \mathcal{S}/\{t_f\},~\forall n_i \in \mathcal{N},~\Sum_{n_j \in
  \mathcal{N}/\{n_i\}} \texttt{edges}[n1][n2][s] + \texttt{edges}[n2][n1][s]
  \leq 1
  \label{eq:cplex_no_swaps}
\end{equation}
Bart\'{a}k and \v{S}ancara noticed that the model is easier to be solved while
using easier constraints and more variables, than while using less variables
and more difficult constraints. This is possible imputable to the fact that the
solver needs to internally transform the constraints into decision variables
and a constraint that is more complex provides more decision variables than a
simpler constraint with a priori defined decision variables. 
%
\subsubsection{Tracking Movements}
Since we need to minimize the objective function, we need to memorize the
movements that are done in the simulation. This is done by using the following 
constraint:
\begin{equation}
  \begin{array}{c}
    \forall s\in\mathcal{S},~\forall a\in\mathcal{A},~\forall
    n_i\in\mathcal{N},~\forall n_j\in\mathcal{N},\\
    X[s-1][n_i][a] \wedge X[s][n_j][a] \Rightarrow \texttt{movement}[a][s] = 
    \texttt{connect}[n_i][n_j] \wedge \texttt{edges}[n_i][n_j][s]
  \end{array}
  \label{eq:movement_track}
\end{equation}
This constraints basically checks if the agent has moved, and if it has, then
it assigns the cost of the edge to the movement matrix.
%
\subsubsection{Goal Reaching}
Another aspect that was not taken into consideration in previous
works~\cite{picat1, picat2} is the absence of intermediate goals that the
agents have to meet before reaching the final destination. For this reason, the
following constraints were added:
\begin{equation}
  \forall a \in \mathcal{A},~\forall s\in\mathcal{S},
  \Sum_{g\in\texttt{n\_goals}[a]} \texttt{goal\_points}[a][g][s] \Rightarrow
  x[s][\texttt{goal\_pos}[a][g]][a]
  \label{eq:take_goals}
\end{equation}
Which means that if the goal has been taken in a given movement $s$ from a
agent $a$, then the agent must be in said position $n$ that time step $s$.
\newline
We also need to make sure that an agent takes at most one goal in each
timestamp and that also each goal is taken at most once:
\begin{equation}
  \forall s\in\mathcal{S},~\forall a \in\mathcal{A},~\Sum_{g\in
  \{1,\hdots,\texttt{n\_goals[a]}\}} \texttt{goal\_points}[a][g][s] \leq 1
  \label{eq:one_goal_one_step}
\end{equation}
\begin{equation}
  \forall a\in\mathcal{A},~\forall g \in\{1,\hdots,\texttt{n\_goals[a]}\}, 
  ~\Sum_{s\in \mathcal{S}} \texttt{goal\_points}[a][g][s] \leq 1
  \label{eq:one_goal_per_life}
\end{equation}
Finally, constraint from Equation~\ref{eq:cplex_final_pos} must be modified in
order to account for the fact that the agents cannot reach the final
destination without having taken the goals:
\begin{equation}
  \forall a\in\mathcal{A} 
  \Sum_{\substack{s\in\mathcal{S}\\g\in\{1,\hdots,\texttt{n\_goals}[a]\}}}
  \texttt{goal\_points}[a][g][s] = \texttt{n\_goals}[a] \wedge X[t_f][E_a][a]
  \label{eq:final_pos}
\end{equation}
%
%
\subsection{Pre-processing, main and Post-processing}
\OPL allows also to have a main function that can be used to create the model,
solve it and analyze whether it was possible to find a feasible solution or
not. This is particularly useful to the \acrs{MAPF} problem since it is not
possible to know a priori which are the correct lengths of the paths to follow.
The idea is to start with a lower bound on the number of steps of the solution
and increase the value each time the solver could not find a solution until one
could be found. Will discuss later with the C++ implementation of a better way
to estimate the lower bound. \newline
The post-processing instead is basically just a way to output information
about the found solution. 
%
%
\subsection{Python and C++ Implementations}
Since CPLEX offers the possibility to integrate the solver in Python or C++, we
tried to develop a more portable version of the software so that it could run
on multiple platforms without requiring the IDE of \OPL. Moreover, having the
possibility to run the constraint programming code in C++ or in Python would
allow us to test the performance far more easily. \newline
The code was first moved to Python resulting in a quite seamless transition
from CPLEX. Once the \textit{docplex} library was imported, the data structures
used to input data are basically the same and the creation of the model is 
straightforward. Also the decision variables have a class that clearly states
that they are used as decision variables reducing the confusion. In same way,
declaring constraints for the model using the given functions is easy. Indeed,
the syntax of the majority of the code stayed almost the same, leading to the
hope that maybe the solver called from Python was the same called from the IDE
of \OPL. Unfortunately we were wrong and the performance dropped dramatically.
\newline
We then decided to move to C++ to check whether the performance could improve.
The CPLEX framework for C++ is called \textit{Concert} and it suffices to say
that moving the code-base to comply with it was not as immediate as it was with
Python. With Concert we first need to describe a \texttt{IloEnv} environment
in which the variables will be stored. We can then create the model and the
CPLEX solver. The types offered by Concert do not have a clear distinction
between decision variables and input (or data) variables. Moreover, in some
cases the name of the types is even misleading, e.g., constraints are
declared with type \texttt{IloRange} that is quite similar to
\texttt{IloIntRange} which is used to define an interval of integers\footnote{
After some time and a little bit of research, one understands that constraints
seem to be extracted as a series of variables and hence the type  
\texttt{IloRange}}, or, for example there is not a function similar to how 
\texttt{sum} works in \OPL IDE or of how \texttt{model.sum()} works in Python,
but instead the sum over a given set of variables must be done by using an
auxiliary variable declared as \texttt{IloExpr} and then incremented in a loop.
The last weak point of Concert is the fact that accessing the variables to
change their values before repeating the solver is not directly possible, at
least to our knowledge Instead one needs to free both the solver, the model and
the environment each time a variable needs to be changed to solve the model
again. That said, moving the code to C++ proved to be a good decision in terms
of performance. \newline
Moreover, doing so allows us to use the shortest path algorithm shown in
Section~\ref{ssec:TDSP} to provide a lower bound in the pre-processing phase of
the program reducing by a lot the number of needed iterations to reach a
solution. \newline
Finally, both the implementations in Python and C++ allow to minimize not only
the \acrl{SIC} function, but also to decide to minimize the makespan. Yet 
again, in Python this is done in a straightforward way by using the function
\texttt{model.max()} to find the maximum length between the paths, while in C++
one needs to create the function manually. To note though, that as in CPLEX, it
is not possible to have two objective functions at once.
