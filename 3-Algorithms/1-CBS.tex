\section{\acrf{CBS}}
\label{sec:cbsAlgo}
As said \cbs is a two level algorithm for solving the \acrs{MAPF} problem in a
correct and optimal way~\cite{MAPF_overview}. The work from Sharon et al. in
which \cbs was first proposed actually contains also a variant of the algorithm
to speed up cases in which two or mare agents are frequently
colliding~\cite{CBS}. This variant is called \acrf{MACBS} and what it does is
to merge the agents that frequently conflicts and consider them as a unique
entity. This though implies the fact that the low-level search must already be
a \acrs{MAPF} solver and not simply a \acrs{SAPF} solver. For this reason, this
thesis focuses on the classical version of \cbs. \newline
We are going to first introduce two algorithms to solve the low-level search of
\cbs and then move to the description of the implementation of the high-level
search.
%
%
\subsection{Low-Level Search}
In \cbs, the low-level search is the algorithm responsible for finding a
feasible path from an initial point to a final point on the graph for a single
agent. We shall distinguish two situations: the first call to the low-level is
for the creation of the root of the \acrf{ct}, while all the other calls are
for subsequent nodes. Each of the nodes, excluding the root, will have
constraints to abide to. The low-level algorithm must be modified to account
for the presence of possible limitations. \newline 
Moreover, the majority of \acrs{SAPF} algorithms are one-shot in the sense that
they do not provide support for additional intermediate goals to reach. As
mentioned in Chapter~\ref{ch:introduction}, this thesis does not consider the
problem of mission planning, but only the motion planning one, this means that
we assume that the order in which the tasks are received is actually the best
one hence the problem resolves to the best way of combining the various
point-to-point paths. \newline
We shall now present two algorithms that have been adapted to solve the
low-level search: the first is based on spanning trees, while the second is a
time dependent shortest path algorithm based on Dijkstra. 
%
\subsubsection{Spanning Tree}
\begin{figure}[tb]
  \centering
  \includegraphics[width=0.35\linewidth]{STExample}
  \caption{A possible example a \acrs{SAPF} problem that the low-level search
  needs to solve. The agent starts from vertex 1 and has to reach vertex 6.}
  \label{fig:STExample}
\end{figure}
The main problem that this algorithm aims at solving is the difficulty in
finding alternative paths when faced with constraints. For example, consider
Figure~\ref{fig:STExample}: it is possible to use two path to go from vertex 1
to vertex 6:
\[
  \begin{array}{l}
    \pi_1 = \{1,2,4,7,6\}\\
    \pi_2 = \{1,2,3,5,8,7,6\}
  \end{array}
\]
Obviously, the second path is longer than the first one and it would never be
selected if the first could be used. Let's suppose though that the agent cannot
be on node 4 at times 3 and 4, then the two paths start to have the same 
length. \newline
\milc{spanningTree} provides in an easy way alternative paths that the
low-level search may select to pass to the high-level search. The algorithm
works as described in Algorithm~\ref{algo:ST}.
\begin{algorithm}[t]
  \DontPrintSemicolon
  \caption{The pseudo code for the spanning tree algorithm}
  \label{algo:ST}

  \KwIn{\texttt{nodes}: the list of nodes}
  \KwIn{\texttt{initPos}: the initial position}
  \KwIn{\texttt{endPos}: the final position}
  \KwIn{\texttt{goalPos}: the list of goal positions}
  \KwIn{\texttt{connect}: the connectivity matrix}
  \;
  \KwData{\texttt{finalPaths}: a vector containing the paths that are valid}
  \;
  \Begin{
    \If{\texttt{goalPos} is empty}{
      \Return \texttt{spanningTreeP2P} between \texttt{initPos} and
      \texttt{endPos}\;
    }
    \Else{
      \texttt{start} $\gets$ \texttt{initPos}\;
      \texttt{end}   $\gets$ \texttt{goalPos[0]}\;
      Insert the solution of \texttt{spanningTreeP2P} from \texttt{start} to
      \texttt{end} in \texttt{finalPath}\;
      \For{Each remaining goal in \texttt{goalPos}}{
        \texttt{start} $\gets$ \texttt{end}\;
        \texttt{end}   $\gets$ next goal\;
        Insert the solution of \texttt{spanningTreeP2P} from \texttt{start} to
        \texttt{end} in \texttt{finalPath}\;
      }

      \texttt{start} $\gets$ last goal\;
      \texttt{end}   $\gets$ \texttt{endPos}\;
      Insert the solution of \texttt{spanningTreeP2P} from \texttt{start} to
      \texttt{end} in \texttt{finalPath}\;
    }
  }
\end{algorithm}
The point-to-point spanning tree function \milc{spanningTreeP2P} works in the
following way:
\begin{enumerate}
  \item Initialize a set of nodes (\texttt{OPEN}) to explore with their
    distance from the source and a list of paths (\texttt{paths}).
  \item Initialize the distance of the first node.
  \item Start from the source, add to \texttt{OPEN} all the neighbors of the 
    source adding one to the distance. 
  \item Then until \texttt{OPEN} has elements, pop the last element from the
    list and iterate with the same procedure as for the source adding the
    neighbors of the considered node to \texttt{OPEN} and adding each
    considered node to the current path. 
  \item At each iteration, the following checks are carried out:
    \begin{itemize}
      \item If the element that was popped is the arrival node, then the path
        we have found until now is valid and can be stored.
      \item If the considered node is already inside the path, then the path is
        a cycle and should not be considered, discard the path.
      \item If the distance of the current node is less than the distance of
        the last node added to the path, then the path is not valid, should be
        dropped and the new node should be added to path.
    \end{itemize}
    Each time one of the conditions is true, either the path is dropped or it
    is stored. In the latter, a new path is added to \texttt{paths} starting
    from the last node that had more than one neighbor added to \texttt{OPEN}.
    When the former happens instead, the path is not added to \texttt{paths},
    but we erase all the nodes up to (and excluded) the node that generated
    more neighbors in \texttt{OPEN}. To check which node is the one responsible
    for multiple neighbors, we check the last distance in \texttt{OPEN} and the
    position of the node to be erased inside its path. 
  \item When \texttt{OPEN} is empty, then we can return. Before returning we
    check if the last path that was being built has reached the final
    destination and if it has not, then it is removed from the list. 
\end{enumerate}
Another key aspect is how to merge the point-to-point paths to obtain a single
path. Indeed this is a weak spot of the algorithm since it is actually a
combinatorial problem which may cause the memory usage to raise exponentially.
Indeed, for every \texttt{spanningTreeP2P} invocation after the first, the
previous vector of solutions is copied as many times as the number of the new
solutions and each one of the solutions is added to the old ones. \newline
The algorithm so described works for solving a \acrs{sapf} problem with the
addition of intermediate goals and it returns a vector of possible paths to
reach the destination while passing through the goals. We now need to also 
consider the possible constraints from high-level. To do this, we first order
the constraints following a temporal order, then we iterate through the paths 
and check whether a given path violates one or more constraints. If it does,
then we insert a copy of a safe node to indicate that the agent must stay on
that node to avoid the conflict. Finally, once we have gone through all the
paths and fixed the possible conflicts, we are going to return to the
high-level the shortest path. \newline
The important point of this problem is how to choose a "safe" location.
Initially, the safe location is the starting node, i.e., index 0, then each
time a constraint is fixed, the safe location moves to the time of the
constraint. This allows to produce a path that is free from collisions. Notice
that it is important that the constraints are temporally ordered, otherwise
adding nodes in which the agent should stay may produce other violations. 
%
\subsubsection{\acrf{TDSP}}
This low-level search is mainly based on the algorithm from 
Dijkstra~\cite{dijkstra}, but we introduce a new structure to handle the edges 
between nodes. \newline
This structure is trivially names \texttt{Connection} and stores two important 
piece of information:
\begin{itemize}
  \item The type of connection, which can either be:
    \begin{itemize}
      \item \texttt{ONE}: when the edge is always valid;
      \item \texttt{ZERO}: when there is never an edge between two nodes;
      \item \texttt{LIMIT\_ONCE}: when the connection is always valid except for
        the specified time step(s);
      \item \texttt{LIMIT\_ALWAYS}: when the connection is never valid except
        for the specified time step(s).
    \end{itemize}
  \item A vector of time steps, which can either be times in which the edge
    should not be considered, or times in which the edge is valid. 
\end{itemize}
Before calling the shortest path algorithm, the low-level search transform the
connectivity matrix from integer values to objects of time \texttt{Connection}
taking into consideration the constraints. This allows the algorithm to avoid a
path if the edge is not valid in that time step as it would lead to a vertex or
a swap conflict. \newline
This alone though may not solve the constraints as all it does is stating that
the agent should or not use a certain edge. In order for the agent to stay on
a node, we need to add a virtual node. We will refer to this virtual node as
placeholder. What the algorithm does is to see that at a given time $t$ the
agent cannot be on a node $n$, then it adds a placeholder for the node, it
limits the connections of all the nodes that previously had a valid connection
to the node $n$ at time $t$ and replace this connections with a valid
connection to the placeholder only for time $t$. This basically increases the
path length of one unit and during the phase of post-processing it is possible
to remove the placeholder and add a node where the agent can stay. \newline
This is the procedure for a vertex conflict. Instead when a swap conflict is
found, we simply consider it as a double vertex conflict: if a node cannot go
from a node $n_i$ to a node $n_j$ at time $t$, this means that at time $t+1$ it
can neither be on $n_i$ since it will be occupied from the other node, nor on
$n_j$ since it would cause a swap conflict. \newline
The rest of the shortest path algorithm is dealt in the following way:
\begin{itemize}
  \item Compute the shortest path between the initial point and the first goal
    point; 
  \item Iterate between the various goal points in list;
  \item Compute the shortest path between the last goal point to the final
    vertex.
\end{itemize}
The so obtained path is the shortest path (assuming correctness and optimality)
that meets all the goals and respects the constraints. \newline
In our case, the shortest path algorithm is the Dijkstra algorithm which is
naturally optimal without having to prove that the used heuristic is
admissible.
%
%
\subsection{High-Level Search}
The high-level search is basically the same as the one described in
Chapter~\ref{ch:motionPlanning}. We start from a root node which does not
contain any constraint and we compute the best path for each agent considering
the agent as if it was in \acrs{sapf} problem. Then we check for conflicts:
\begin{itemize}
  \item if no conflict was found, then this is the solution to return;
  \item if a one or more conflicts were found, then only the first conflict is
    considered. 
\end{itemize}
To check for conflicts, we compare each path with all the other paths. Each
element in the vector corresponds to the node occupied by an agent at a certain
time $t$, which is actually the index of the node inside the vector. So when a
vertex conflict is found, it happens because at the same index, both vectors
have the same node, while a swap conflict is found when then node of a vector
at an index $t$ corresponds to the node at index $t+1$ of the other vector and
vice versa. \newline
When a conflict is found, two constraints are produced:
\begin{itemize}
  \item Vertex conflict: in one constraint, agent $a_i$ cannot be on node $n$
    at time $t$, on the second constraint, agent $a_j$ cannot be on node $n$ at
    time $t$.
  \item Swap conflict: in one constraint, agent $a_i$ cannot go from node $n_1$
    to node $n_2$ at time $t$, on the second constraint, agent $a_j$ cannot go
    from node $n_2$ to node $n_1$ at time $t$. 
\end{itemize}
For each constraint, the low-level search is called: since the constraint
affects only one agent, only its path is going to be recomputed. If the
low-level could find a solution, then the high-level adds a new node with the
new solution and also adds the constraint to the constraints list of the new
node. The search is repeated until a valid solution is found. 
%TODO we have not talked about how to move from weighted edges to unitary edges
